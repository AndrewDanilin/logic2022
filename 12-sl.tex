\documentclass[aspectratio=169]{beamer}
\usepackage[utf8]{inputenc}
\usepackage[english,russian]{babel}
\usepackage{cancel}
\usepackage{amssymb}
\usepackage{stmaryrd}
\usepackage{cmll}
\usepackage{graphicx}
\usepackage{amsthm}
\usepackage{tikz}
\usepackage{multicol}
\usetikzlibrary{patterns}
\usepackage{chronosys}
\usepackage{proof}
\usepackage{multirow}
\setbeamertemplate{navigation symbols}{}
%\usetheme{Warsaw}

\newtheorem{thm}{Теорема}[section]
\newtheorem{dfn}{Определение}[section]
\newtheorem{lmm}{Лемма}[section]
\newtheorem{exm}{Пример}[section]
\newtheorem{snote}{Пояснение}[section]

\newcommand{\divisible}%                                                     
{\mathrel{\lower.2ex%
\vbox{\baselineskip=0.7ex\lineskiplimit=0pt%
\kern6pt \hbox{.}\hbox{.}\hbox{.}}%
}}

\begin{document}

\newcommand\doubleplus{+\kern-1.3ex+\kern0.8ex}
\newcommand\mdoubleplus{\ensuremath{\mathbin{+\mkern-10mu+}}}

\begin{frame}{}
\LARGE\begin{center}Теорема Лёвенгейма-Сколема\end{center}
\end{frame}

\begin{frame}{Как пересчитать вещественные числа (неформально)?}
\begin{enumerate}
\item Номер вещественного числа --- первое упоминание в литературе, т.е. $\langle j, y, n, p, r, c \rangle$:\\
j --- гёделев номер названия научного журнала (книги);\\
y --- год издания;\\
n --- номер;\\
p --- страница;\\
r --- строка;\\
c --- позиция\pause
\item Попробуете предъявить число $x$, не имеющее номера? Это рассуждение сразу даст номер.\\
\end{enumerate}
\end{frame}

\begin{frame}{Мощность модели и аксиоматизации}
\begin{dfn} Пусть задана модель $\langle D, F_n, P_n \rangle$ для некоторой теории первого порядка. 
Её мощностью будем считать мощность $D$.
\end{dfn}\pause

\begin{dfn} Пусть задана формальная теория с аксиомами $\alpha_n$. Её мощность --- мощность множества $\{\alpha_n\}$.
\end{dfn}\pause

\begin{exm} Формальная арифметика, исчисление предикатов, исчисление высказываний --- счётно-аксиоматизируемые.
\end{exm}
\end{frame}

\begin{frame}{Элементарная подмодель}
\begin{dfn}$\mathcal{M}' = \langle D', F'_n, P'_n \rangle$ --- элементарная подмодель $\mathcal{M} = \langle D, F_n, P_n \rangle$, 
если: \pause
\begin{enumerate}
\item $D' \subseteq D$, \pause $F'_n$, $P'_n$ --- сужение $F_n$, $P_n$ (замкнутое на $D'$). \pause
\item $\mathcal{M}\models \varphi(x_1,\dots,x_n)$ тогда и только тогда, когда $\mathcal{M}'\models \varphi(x_1,\dots,x_n)$
при $x_i \in D'$. \pause
\end{enumerate}
\end{dfn}

\begin{exm}Когда сужение $M$ не является элементарной подмоделью? \pause

$\forall x.\exists y.x \ne y$. Истинно в $\mathbb{N}$. \pause Но пусть $D' = \{ \square \}$.
\end{exm}
\end{frame}

\begin{frame}{Теорема Лёвенгейма-Сколема}
\begin{thm}Пусть $T$ --- множество всех формул теории первого порядка. 
Пусть теория имеет некоторую модель $\mathcal{M}$.
Тогда найдётся элементарная подмодель $\mathcal{M'}$, причём $|\mathcal{M'}| = \max(\aleph_0, |T|)$.
\end{thm}\pause

\begin{proof} (Схема доказательства)
\begin{enumerate} 
\item Построим $D_0$ --- множество всех значений, которые упомянуты в языке теории. \pause
\item Будем последовательно пополнять $D_i$: $D_0 \subseteq D_1 \subseteq D_2 \dots$, следя за мощностью.
$D' = \cup D_i$.
\item Покажем, что $\langle D', F_n, P_n\rangle$ --- требуемая подмодель.
\end{enumerate}
\end{proof}
\end{frame}

\begin{frame}{Начальный $D_0$}
Пусть $\{f^0_k\}$ --- все 0-местные функциональные символы теории. \pause
\begin{enumerate}
\item $D_0 = \{ \llbracket f^0_k \rrbracket \}$, если есть хотя бы один $f^0_k$. \pause
\item Если таких $f^0_k$ нет, возьмём какое-нибудь одно значение из $D$. \pause
\end{enumerate}\pause

Очевидно, $|D_0| \le |T|$.
\end{frame}

\begin{frame}{Пополнение $D$}
Фиксируем некоторый $D_k$. Напомним, $T$ --- множество всех формул теории. Рассмотрим $\varphi \in T$.\pause
\begin{enumerate}
\item $\varphi$ не имеет свободных переменных --- пропустим. \pause
\item $\varphi$ имеет хотя бы одну свободную переменную $y$. \pause
\begin{enumerate}
\item $\varphi (y, x_1, \dots, x_n)$ при $y,x_i \in D_k$ бывает истинным и ложным --- ничего не меняем \pause
\item $\varphi (y, x_1, \dots, x_n)$ при $y \in D$ и $x_i \in D_k$ либо всегда истинен, либо всегда ложен --- ничего не меняем \pause
\item $\varphi (y, x_1, \dots, x_n)$ при $y,x_i \in D_k$ тождественно истинен или ложен, но при 
$y' \in D \setminus D_k$ отличается --- добавим $y'$ к $D_{k+1}$. \pause
Вместе добавим всевозможные $\llbracket\theta(y')\rrbracket$.
\end{enumerate}
\end{enumerate}\pause

Всего добавили не больше $|T| \cdot |D_k|$. \pause $|\cup D_i| \le |T| \cdot |D_k| \cdot |\aleph_0| = \max (|T|, |\aleph_0|)$
\end{frame}

\begin{frame}{$\mathcal{M}'$ --- элементарная подмодель}
Индукцией по структуре формул $\tau \in T$ покажем, 
что все формулы можно вычислить, и что $\llbracket \varphi \rrbracket_\mathcal{M'} = \llbracket \varphi \rrbracket_\mathcal{M}$.\pause

\begin{enumerate}
\item База, 0 связок. $\tau \equiv P(f_1(x_1,\dots,x_n),\dots,f_n(x_1,\dots,x_n))$. \pause Если $x_i \in D'$, то значит,
добавлены на некоторых шагах (максимальный пусть $t$). Поэтому в $D_{t+1}$ можно вычислить формулу, и её значение сохранилось. \pause
\item Переход. Пусть формулы из $k$ связок сохраняют значения. Рассмотрим $\tau$ с $k+1$ связкой. \pause
\begin{enumerate}
\item $\tau \equiv \rho \star \sigma$ --- очевидно. \pause
\item $\tau\equiv\forall y.\varphi(y,x_1,\dots,x_n)$. \pause 
Каждый $x_i$ добавлен на каком-то шаге --- максимум $t$. \pause 
Если $\varphi(y,x_1,\dots,x_n)$ бывает истинен и ложен при $y_t, y_f \in D$, то $y_t, y_f \in D_{t+1}$ (по построению). \pause
Поэтому, если $\mathcal{M}\not\models\forall y.\varphi(y,x_1,\dots,x_n)$, то и 
$\mathcal{M'}\not\models\forall y.\varphi(y,x_1,\dots,x_n)$. \pause
Если же $\varphi(y,x_1,\dots,x_n)$ не меняется от $y$, то тем более
$\llbracket \varphi \rrbracket_\mathcal{M'} = \llbracket \varphi \rrbracket_\mathcal{M}$. \pause
\item $\tau\equiv\exists y.\varphi(y,x_1,\dots,x_n)$ --- аналогично.
\end{enumerate}
\end{enumerate}
\end{frame}

\begin{frame}{<<Парадокс>> Сколема}
\begin{enumerate}
\item Как известно, $|\mathbb{R}| = |\mathcal{P}(\mathbb{N})| > |\mathbb{N}| = \aleph_0$. \pause Однако, ZFC --- счётно-аксиоматизируемая теория. \pause
Значит, существует счётная модель ZFC, то есть $|\mathbb{R}| = \aleph_0$. \pause В чём ошибка? \pause
\item У равенств разный смысл, первое --- в предметном языке, второе --- в метаязыке. \pause Внутри теории не выразить все 
способы нумерации, которые возможны.
\end{enumerate}
\end{frame}

\end{document}
