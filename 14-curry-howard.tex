\documentclass[aspectratio=169]{beamer}
\usepackage[utf8]{inputenc}
\usepackage[english,russian]{babel}
\usepackage{cancel}
\usepackage{amssymb}
\usepackage{stmaryrd}
\usepackage{cmll}
\usepackage{graphicx}
\usepackage{amsthm}
\usepackage{tikz}
\usepackage{multicol}
\usetikzlibrary{patterns}
\usepackage{chronosys}
\usepackage{proof}
\usepackage{multirow}
\setbeamertemplate{navigation symbols}{}
%\usetheme{Warsaw}

\newtheorem{thm}{Теорема}[section]
\newtheorem{dfn}{Определение}[section]
\newtheorem{lmm}{Лемма}[section]
\newtheorem{exm}{Пример}[section]
\newtheorem{snote}{Пояснение}[section]

\newcommand{\divisible}%
{\mathrel{\lower.2ex%
\vbox{\baselineskip=0.7ex\lineskiplimit=0pt%
\kern6pt \hbox{.}\hbox{.}\hbox{.}}%
}}

\begin{document}

\newcommand\doubleplus{+\kern-1.3ex+\kern0.8ex}
\newcommand\mdoubleplus{\ensuremath{\mathbin{+\mkern-10mu+}}}

\begin{frame}{}
\begin{center}\Large Изоморфизм Карри-Ховарда \end{center}
\end{frame}

\begin{frame}{Лямбда-исчисление}
Синтаксис:
$$\Lambda ::= (\lambda x.\Lambda) | (\Lambda\ \Lambda) | x$$

\pause

Интуиция: вызов функции.
\begin{center}\begin{tabular}{l|l}
$\lambda$-выражение & Python \\\hline
$\lambda f.\lambda x.f\ x$ & \texttt{def one(f,x): return f(x)}\\
$(\lambda x.x\ x)\ (\lambda x.x\ x)$ & \texttt{(lambda x: x x) (lambda x: x x)}\\
 &   \texttt{def omega(x): return x(x); omega(omega)}
\end{tabular}\end{center}

\pause

Исчисление: изучаем преобразования формул --- редукция.

$$(\lambda x.A)\ B \rightarrow_\beta A [x := B]$$

\begin{exm}
$(\lambda x.x\ x)\ (\lambda n.n) \rightarrow_\beta (\lambda n.n)\ (\lambda n.n) \rightarrow_\beta \lambda n.n$\\
$(\lambda x.x\ x)\ (\lambda x.x\ x) \rightarrow_\beta (\lambda x.x\ x)\ (\lambda x.x\ x)$
\end{exm}
\end{frame}

\begin{frame}{Чёрчевские нумералы}
$$f^{(n)}(x) = \left\{\begin{array}{ll}x, & n = 0\\f(f^{(n-1)}(x)), & n > 0\end{array}\right.$$

\begin{dfn}
Чёрчевский нумерал $\overline{n} = f^{(n)}(x)$
\end{dfn}

\begin{exm}
$\overline{3} = \lambda f.\lambda x.f(f(f(x)))$

Инкремент: $Inc = \lambda n.\lambda f.\lambda x.n\ f\ (f\ x)$
\end{exm}\vspace{-0.3cm}

$$\begin{array}{l}(\lambda n.\lambda f.\lambda x.n\ f\ (f\ x))\ \overline{0} = (\lambda n.\lambda f.\lambda x.n\ f\ (f\ x))\ (\lambda f'.\lambda x'.x') \rightarrow_\beta \pause\\
  \dots\lambda f.\lambda x.(\lambda f'.\lambda x'.x')\ f\ (f\ x) \rightarrow_\beta \pause\\
  \dots\lambda f.\lambda x.(\lambda x'.x')\ (f\ x) \rightarrow_\beta \pause\\
  \dots\lambda f.\lambda x.f\ x = \overline{1}\end{array}$$
\pause

Декремент: $Dec = \lambda n.\lambda f.\lambda x.n\ (\lambda g.\lambda h.h\ (g\ f))\ (\lambda u.x)\ (\lambda u.u)$
\end{frame}

\begin{frame}{Просто-типизированное лямбда-исчисление}

\begin{dfn}Импликационный фрагмент интуиционистской логики:

$$\infer{\Gamma,\varphi \vdash \varphi}{} \quad\quad 
  \infer{\Gamma\vdash\varphi\rightarrow\psi}{\Gamma,\varphi\vdash\psi} \quad\quad 
  \infer{\Gamma\vdash\psi}{\Gamma\vdash\varphi\quad\quad\Gamma\vdash\varphi\rightarrow\psi}$$
\end{dfn}

\begin{dfn}
Просто-типизированное лямбда-исчисление. \pause Типы: $\tau ::= \alpha | (\tau\rightarrow\tau)$. \pause Язык: $\Gamma\vdash A:\varphi$
$$\infer[x \notin \Gamma]{\Gamma,x:\varphi \vdash x:\varphi}{} \quad\quad 
  \infer[x \notin \Gamma]{\Gamma\vdash \lambda x.A: \varphi\rightarrow\psi}{\Gamma,x:\varphi\vdash A:\psi} \quad\quad 
  \infer{\Gamma\vdash B A:\psi}{\Gamma\vdash A:\varphi\quad\quad\Gamma\vdash B:\varphi\rightarrow\psi}$$
\end{dfn}
\end{frame}

\begin{frame}{Изоморфизм Карри-Ховарда}

\begin{tabular}{ll}
$\lambda$-исчисление & исчисление высказываний\\\hline
Выражение & доказательство\\
Тип выражения & высказывание\\
Тип функции & импликация \\
Упорядоченная пара & конъюнкция \\
Алгебраический тип & дизъюнкция
\end{tabular}

\end{frame}

\begin{frame}{Изоморфизм Карри-Ховарда: отрицание}

\begin{dfn}Ложь ($\bot$) --- необитаемый тип;
$\texttt{failwith/raise/throw} : \alpha\rightarrow\bot$; $\neg\varphi\equiv\varphi\rightarrow\bot$
 \end{dfn}

Например, контрапозиция:
$(\alpha\rightarrow\beta)\rightarrow(\neg\beta\rightarrow\neg\alpha)$

$$\infer[\lambda]{\lambda f^{\alpha\rightarrow\beta}.\lambda n^{\beta\rightarrow\bot}.\lambda a^\alpha.n\ (f\ a): (\alpha\rightarrow\beta)\rightarrow(\neg\beta\rightarrow\neg\alpha)}
        {\infer[\lambda]{f:\alpha\rightarrow\beta\vdash\lambda n^{\beta\rightarrow\bot}.\lambda a^\alpha.n\ (f\ a): \neg\beta\rightarrow\neg\alpha}
               {\infer[\lambda]{f:\alpha\rightarrow\beta,n:\beta\rightarrow\bot\vdash\lambda a^\alpha.n\ (f\ a): \neg\alpha}{
                       \infer[App]{f:\alpha\rightarrow\beta,n:\beta\rightarrow\bot, a:\alpha \vdash n\ (f\ a): \bot}{
                              \infer[App]{\Phi \vdash f\ a: \beta}{\infer[Ax]{\Phi \vdash f:\alpha\rightarrow\beta}{}\quad\quad \infer[Ax]{\Phi \vdash a: \alpha}{}}
    \quad\quad \infer[Ax]{\Phi \vdash n: \beta\rightarrow\bot}{}
               }
               }}}$$

Снятие двойного отрицания: $((\alpha\rightarrow\bot)\rightarrow\bot)\rightarrow\alpha$, то есть $\lambda f^{(\alpha\rightarrow\bot)\rightarrow\bot}.?: \alpha$.
\end{frame}

\begin{frame}{Исчисление второго порядка}
\begin{itemize}
\item Напомним о порядках:

\begin{tabular}{lll}
Порядок & Объекты & Пример\\\hline
0 (И.В.) & Атомарные & $P$ \\
1 (И.П. 1) & Множества & $\{ x | P(x) \}$ \\
2 (И.П. 2) & Множества множеств & $\{ P | \forall t.t > 0 \rightarrow P(t) \}$
\end{tabular}\pause

\item Можно заменить схемы аксиом на аксиомы:
$\forall a.\forall b.a \rightarrow b \rightarrow a$\pause

\item
Не так просто! $W(x) := \forall p.\forall x.p(x)$, рассматриваем ли
$\llbracket \forall x.p(x) \rrbracket^{p := W(x)}$? (импредикативность)
\pause

\item
Простой вариант: только пропозициональные переменные. $\forall p.p \rightarrow p$ \pause
$$\llbracket\forall p.Q\rrbracket = \left\{\begin{array}{ll}\text{И}, & 
         \llbracket Q \rrbracket^{p := \text{И}} = \llbracket Q \rrbracket^{p := \text{Л}} = \text{И}\\
        \text{Л}, & \text{иначе}\end{array}\right.$$


\end{itemize}
\end{frame}

\begin{frame}{Изоморфизм Карри-Ховарда для логики второго порядка}
Типы и значения, зависящие от типов.
\begin{itemize}
\item Что такое $T: \forall x.x \rightarrow x$? \pause \\
 \texttt{template <class x> class T \{ x f (x); \}} \pause
\item Что такое $T: \exists x.\tau(x)$? \pause \\
 Абстрактный тип данных: $\texttt{interface T \{$\tau$\}; f(T x)}$
\end{itemize}
\end{frame}

\begin{frame}{Зависимые типы}
\begin{itemize}\item Рассмотрим код\\ \texttt{int n; cin >{}>{} n; int arr[n];} \\
Каков тип \texttt{arr}? \pause
\item $\texttt{sizeof(arr)} = n \cdot \texttt{sizeof(int)}$ \pause
\item $arr = \Pi n^{\texttt{int}}.\texttt{int[}n\texttt{]}$ \pause
\item Аналогично, \texttt{printf(const char*, ...)} --- капитуляция. \pause
\item Есть языки, где тип выписывается (например, Идрис).
\end{itemize}
\end{frame}

\begin{frame}{Прямолинейное: доказательства в коде}
\begin{itemize}
\item \texttt{Div2: (l: int) -> (even l) -> int}\pause
\item \texttt{even l} --- что это?\pause
\item $$even (x) ::= \left\{\begin{array}{ll} EZ, & x = 0\\ EP(even(y)), & x = y''\end{array}\right.$$\pause
\item \texttt{Div2 10 (EP (EP (EP (EP (EP EZ)))))}\pause
\item А если \texttt{Div2 p}? В общем случае сложно.

\texttt{Plus2: (l: int) -> (p: even l) -> (l+2, even (l+2)) = (l+2, EP p)}
\end{itemize}
\end{frame}

\begin{frame}{Интереснее: доказательства утверждений}

Натуральные числа: $\texttt{Nat} ::= 0 | \texttt{suc Nat}$,

$$a + b = \left\{\begin{array}{ll} a, & b = 0\\ \texttt{suc }(a + c), & b = \texttt{suc }c\end{array}\right.$$

\texttt{\\func pmap {A B : \\Type} (f : A -> B) \{a a' : A\} (p : a = a') : f a = f a' => ...}

\texttt{\\func +-comm (n m : Nat) : n + m = m + n\\
  | 0, 0 => idp\\
  | suc n, 0 => pmap suc (+-comm n 0)\\
  | 0, suc m => pmap suc (+-comm 0 m)\\
  | suc n, suc m => pmap suc (+-comm (suc n) m *> \\
                    pmap suc (inv (+-comm n m)) *> +-comm n (suc m))}

\end{frame}

\begin{frame}{Гомотопическая теория типов}
\begin{dfn}Изоморфизм Карри-Ховарда-Воеводского.

\begin{tabular}{lll}
Логика & $\lambda$-исчисление & Топология\\\hline
Утверждение & Тип & Пространство \\
Доказательство & Значение & Точка в пространстве\\
Предикат $(=)$ & Зависимый тип $(=)$ & Путь между точками
\end{tabular}\end{dfn}

\begin{enumerate}
\item Точный смысл равенства.
\item Позволяет легко формулировать утверждения про топологию, гомологическую алгебру и т.п.
\item Можно реализовать (кубическая теория типов). Реализации для Агды, Кока, ..., отдельные языки (Аренд)
\end{enumerate}
\end{frame}

\begin{frame}{Какие вопросы пытаемся решать?}
\begin{exm}Самое простое: $x=y$. Почему $x^2 = y^2$?\end{exm} \pause

А что если так $(a = b) = \{ \langle a,b \rangle | a < 10 \with b < 10 \}$?
Тогда $5=7$, но $25 \ne 49$. \pause

Постулируется в формальной арифметике: $(A2)\ a = b \rightarrow a' = b'$ \pause

\begin{proof}Путь $x$ в $y$ --- функция $f: [0,1] \rightarrow S$, 
$f(0)=x$, $f(1)=y$. $f(x) = x^2$ --- непрерывная функция. Тогда $f(x^2)$ --- тоже непрерывная,
то есть $x^2 = y^2$.\end{proof}
\end{frame}

\begin{frame}{Что ещё}
\begin{itemize}
\item Метод резолюций и рядом --- Prolog, SMT-солверы,...
\item Можно пытаться совмещать (F*, ...)
\end{itemize}
\end{frame}
\end{document}
